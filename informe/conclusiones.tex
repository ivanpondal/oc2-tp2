Este trabajo nos permitió ver en casos reales las ventajas del procesamiento de
datos en simultáneo. A su vez pudimos comparar qué tanto mejora el realizar una
implementación en ASM contra una en C, logrando así tener una mejor idea de
cuándo realmente conviene decidirse por uno o por el otro.

Por un lado vimos que en lo que era \textit{Diferencia de Imágenes}, la
implementación en ASM superó en forma desmedida la de C incluso con
optimzaciones del compilador. Sin embargo con \textit{Blur Gaussiano} esta
diferencia no fue tan abismal para la implementación de control, e incluso con
la versión en C mejorada para utilizar mejor la matriz de convolución pudimos
observar cómo nuestra versión en ASM se quedaba atrás. Esto refleja cómo algunas
implementaciones en C el compilador es capaz de optimizar mejor que otras,
llevándonos a repensar cómo mejorar nuestros programas.

En lo que respecta la precisión contra el rendimiento en el algoritmo de
\textit{Blur Gaussiano}, los resultados fueron en parte alentadores, porque
vimos cómo se reducía el tiempo de cómputo, pero a su vez nos mostró que a veces
la precisión es un factor que afecta considerablemente el resultado, como lo fue
en nuestro caso.

Por último, quedan muchos experimentos más que se podrían haber realizado así
como profundizado. En particula, con la cuestión de precisión contra
rendimiento, se podrían haber realizado un mayor número de pruebas, variando los
parámetros de entrada para así poder observar con mayor detenimiento si era en
todos los casos que la imagen resultante se veía afectada, o sólo en cierto
rango. A su vez, se podría mejorar el código para que en lugar de trabajar con
unsigned shorts, lo hiciera con unsigned bytes, nuevamente duplicando la
cantidad de pixeles a procesar en simultáneo.
