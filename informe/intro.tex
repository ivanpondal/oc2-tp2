El objetivo de este trabajo fue estudiar e implementar en una aplicación real el
modelo de procesamiento SIMD (Single Instruction Multiple Data). Para esto se
nos encomendó la tarea de desarrollar en C y ASM (Intel64) dos filtros de
imágenes con los cuales debíamos por un lado analizar cómo competían las
implementaciones y por otro lado realizar un experimento adicional en base al
trabajo ya hecho con el fin de profundizar el estudio del mismo.

Los filtros en cuestión fueron \textit{Diferencia de Imágenes} y \textit{Blur
Gaussiano}. El primero, dado dos imágenes del mismo tamaño, genera una imagen de salida
en escala de grises donde las partes similares o iguales entre las dos entradas
se verán oscuras o negras, mientras que las que difieran, aparecerán más
blancas. El segundo, dada una imagen, un coeficiente real $\sigma$ y un \textit{r}
entero, genera una imágen desenfocada en base a los parámetros de entrada.

Para \textit{Diferencia de Imágenes}, se realizó el estudio de rendimiento entre
la implementación en C y la de ASM utilizando SIMD. En lo que respecta
\textit{Blur Gaussiano}, también se hizo la prueba de performance entre
implementaciones con la diferencia de que en C realizamos dos códigos distintos
donde uno presenta una optimización vinculada con la forma en la que calcula
la convolución necesaria para el efecto. A su vez, decidimos experimentar con el
balance entre performance y precisión del filtro, para lo cual desarrollamos
una segunda implementación en ASM con la que buscamos procesar más datos en
simultáneo sacrificando precisión en las cuentas.
